% Author: Till Tantau
% Source: The PGF/TikZ manual
\documentclass[a4paper,11pt]{article}
\usepackage[utf8]{inputenc}
\usepackage{listings}
\usepackage{amsmath}    % need for subequations
\usepackage{graphicx}   % need for figures
\usepackage{verbatim}   % useful for program listings
\usepackage{color}      % use if color is used in text
%\usepackage{subfigure}  % use for side-by-side figures
\usepackage{hyperref}   % use for hypertext links, including those to external documents and URLs
\usepackage{url}
\usepackage{float}
\usepackage{todonotes}
\usepackage{tikz}
\usepackage{enumitem}
\usepackage{hyperref}
\usepackage{pdfpages}
\usepackage{caption}
\usepackage{subcaption}
\usepackage{listings}
\usepackage{color}
\usepackage{amsfonts}
\usepackage{latexsym}
\usepackage[T1]{fontenc} % use for allowing < and > in cleartext
\usepackage{fixltx2e}    % use for textsubscript
\usepackage[linesnumbered,boxed,ruled]{algorithm2e}
% \newcommand{\BigO}[1]{\ensuremath{\operatorname{O}\left(#1\right)}}
\newcommand{\BigO}[1]{\ensuremath{\mathop{}\mathopen{}\mathcal{O}\mathopen{}\left(#1\right)}}
\graphicspath{ {./images/} }
\definecolor{mygreen}{rgb}{0,0.6,0}
\definecolor{mygray}{rgb}{0.5,0.5,0.5}
\definecolor{mymauve}{rgb}{0.58,0,0.82}
\lstset{ %
  backgroundcolor=\color{white},   % choose the background color; you must add \usepackage{color} or \usepackage{xcolor}
  basicstyle=\footnotesize,        % the size of the fonts that are used for the code
  breakatwhitespace=false,         % sets if automatic breaks should only happen at whitespace
  breaklines=true,                 % sets automatic line breaking
  captionpos=b,                    % sets the caption-position to bottom
  commentstyle=\color{mygreen},    % comment style
  deletekeywords={...},            % if you want to delete keywords from the given language
  escapeinside={\%*}{*)},          % if you want to add LaTeX within your code
  extendedchars=true,              % lets you use non-ASCII characters; for 8-bits encodings only, does not work with UTF-8
  %frame=single,                    % adds a frame around the code
  keepspaces=true,                 % keeps spaces in text, useful for keeping indentation of code (possibly needs columns=flexible)
  keywordstyle=\color{blue},       % keyword style
  language=Octave,                 % the language of the code
  morekeywords={*,...},            % if you want to add more keywords to the set
  numbers=left,                    % where to put the line-numbers; possible values are (none, left, right)
  numbersep=5pt,                   % how far the line-numbers are from the code
  numberstyle=\tiny\color{mygray}, % the style that is used for the line-numbers
  rulecolor=\color{black},         % if not set, the frame-color may be changed on line-breaks within not-black text (e.g. comments (green here))
  showspaces=false,                % show spaces everywhere adding particular underscores; it overrides 'showstringspaces'
  showstringspaces=false,          % underline spaces within strings only
  showtabs=false,                  % show tabs within strings adding particular underscores
  stepnumber=2,                    % the step between two line-numbers. If it's 1, each line will be numbered
  stringstyle=\color{mymauve},     % string literal style
  tabsize=2,                       % sets default tabsize to 2 spaces
  %title=\lstname                   % show the filename of files included with \lstinputlisting; also try caption instead of title
}

\bibliographystyle{plain}
\begin{document}
\date{TODO DATO}
\title{Eye Tracking\\SIGB Spring 2014}

\author{Marcus Gregersen\\
\texttt{mabg@itu.dk} 
\and Martin Faartoft\\
\texttt{mlfa@itu.dk}
\and Mads Westi\\
\texttt{mwek@itu.dk}}
%TODO vejleder og institut
\clearpage\maketitle
\thispagestyle{empty}
\setcounter{page}{1}
\newpage

\section{Introduction}

%billede af øje med navne på de forskellige ting

\section{Pupil Detection}
In this section, we will investigate and compare different techniques for pupil detection.

\subsection{Thresholding}
An obvious first choice of technique, is using a simple threshold to find the pupil, then do connected component (blob) analysis, and finally fit an ellipse on the most promising blobs.

Fig \ref{fig:eye1_threshold_93} shows an example of an image from the 'eye1.avi' sequence and the binary image produced by, using a threshold that blacks out all pixels with intensities above 93. This manages to separate the pupil nicely from the iris. 

\begin{figure}[ht]
  \centering
  \includegraphics[scale=0.2]{eye1}
  \includegraphics[scale=0.2]{eye1_threshold_93}
  \caption{Thresholding eye1.avi}
  \label{fig:eye1_threshold_93}
\end{figure}

The next step, is to do connected component analysis, and fit an ellipsis through the blobs. As seen in fig \ref{fig:eye1_unfiltered}, this succesfully detects the pupil, but is extremely sensitive to noise.

\begin{figure}[ht]
  \centering
  \includegraphics[scale=0.3]{eye1_unfiltered}
  \caption{Fitting ellipses on blobs from eye1.avi (green figures are ellipses fitted through blobs, red dots are the centerpoint of each blob)}
  \label{fig:eye1_unfiltered}
\end{figure}

By experimenting, we find that requiring that the area of the blob lies in the interval $[1000:10000]$, and the extent between $[0.4:1.0]$, we eliminate most false positives on the entire eye1 sequence, while still keeping the true positive.

\paragraph{}
This approach has several problems, however. Note how the true positive on fig \ref{fig:eye1_unfiltered} fails to follow the bottom of pupil correctly. This is due to the glints obscuring part of the boundary between pupil and iris. It also makes some sweeping assumptions:

\paragraph{The pupil has size at least size 1000}
If the person on the sequence leans back slightly, the pupil will shrink and we will fail to detect it.

\paragraph{A threshold of 93 will cleanly separate pupil from iris}
This is true for eye1.avi, but generalizes extremely poorly to the other sequences. If this approach is to be used across multiple sequences recorded in different lighting conditions, the threshold will have to be adjusted by hand for each one.

This problem can be mitigated somewhat. By using Histogram Equalization, a threshold of 25 fares considerably better across several sequences. Note that this will still fail, if parts of the image are significantly darker than the pupil.

\begin{figure}[ht]
  \centering
  \includegraphics[scale=0.2]{eye1}
  \includegraphics[scale=0.2]{eye1_hist_eq}
  \caption{Eye1 before and after Histogram Equalization}
  \label{fig:eye1_hist_eq}
\end{figure}

\begin{figure}[ht]
  \centering
  \includegraphics[scale=0.2]{eye3}
  \includegraphics[scale=0.2]{eye3_hist_eq}
  \caption{Eye3 before and after Histogram Equalization}
  \label{fig:eye3_hist_eq}
\end{figure}


%current:
%threshold->contour detection->contour filtering->ellipsis detection

%clustering (part 2)

%what assumptions do we make?
%how does it perform, when does it work, what causes it to fail
%test on own sequence

%improvements:
%Hough-transform for ellipsis fitting/detection
%histogram equalization/normalization for making thresholding more robust across sequences
%experiments with morphology


\section{Glint Detection}
%current:
%threshold->morphology->contour detection->contour filtering->distance from pupil center

%what assumptions do we make?
%how does it perform, when does it work, what causes it to fail
%test on own sequence

%improvements:
%Laplacian?
%hist equal?

\section{Eye Corner Detection}

%what assumptions do we make?
%how does it perform, when does it work, what causes it to fail
%test on own sequence

\section{Iris / Limbus Detection}

%what assumptions do we make?
%how does it perform, when does it work, what causes it to fail
%test on own sequence


\section{Conclusion}
 
\newpage

\begin{thebibliography}{}

\bibitem{paper:foo}
Foo
      %Sådan her ref'er man en URL
      %\bibitem{lit:json}
      %\url{http://tools.ietf.org/html/rfc4627}
      %Retrieved: 2013-05-02
\end{thebibliography}

%code in appendix
\section*{Appendix}
%\lstinputlisting[language=Python]{../tools/size_estimator.py}


\end{document}
